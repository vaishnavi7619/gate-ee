\let\negmedspace\undefined
\let\negthickspace\undefined
\documentclass[a4,12pt,onecolumn]{IEEEtran}
\usepackage{amsmath,amssymb,amsfonts,amsthm}
\usepackage{algorithmic}
\usepackage{graphicx}
\usepackage{textcomp}
\usepackage{xcolor}
\usepackage{txfonts}
\usepackage{listings}
\usepackage{enumitem}
\usepackage{mathtools}
\usepackage{gensymb}
\usepackage[breaklinks=true]{hyperref}
\usepackage{tkz-euclide}
\usepackage{listings}
\usepackage{circuitikz}
\usepackage{gvv}
\title{Signals and Systems - Gate2023-EE-Q46}
\author{EE23BTECH11014- Devarakonda Guna vaishnavi}

\begin{document}
\maketitle
\textbf{Question}
Consider the state-space description of an LTI system with matrices
\begin{align*}
A =  \myvec{0 & 1 \\ -1 & -2}, B=\myvec {0 \\ 1}, C =\myvec {3 & -2}, D= \myvec{1}.
\end{align*}

For the input, $\sin(\omega t)$, $\omega > 0$, the value of $\omega$ for which the steady-state output of the system will be zero, is \underline{\hspace{2cm}} (Round off to the nearest integer).\\
\hfill(GATE EE 2023)\\
\solution\\
\begin{table}[h!]
    \centering
    \begin{tabular}{|c|c|}
    \hline
    \textbf{Parameter} & \textbf{Value} \\
    \hline
    System Matrix, \(A\) & 
    \(
    \myvec{
        0 & 1 \\
        -1 & -2
    }
    \) \\
    \hline
    Input Matrix, \(B\) & 
    \(
    \myvec{
        0 \\
        1
    }
    \) \\
    \hline
    Output Matrix, \(C\) & 
    \(
    \myvec{
        3 & -2
    }
    \) \\
    \hline
    Feedthrough Matrix, \(D\) & \( \myvec{
    1}
    \) \\
    \hline
    Input Signal, \(u(t)\) & \(\sin(\omega t)\), \(\omega > 0\) \\
    \hline
\end{tabular}
    \caption{Input Parameters}
    \label{table:parameters}
\end{table}
The state-space representation of the system is given by:
\begin{align}
\dot{x}(t) &= Ax(t) + Bu(t) \\
y(t) &= Cx(t) + Du(t)
\end{align}
Transfer function given by:
\begin{align}
T.F &= C\myvec {sI-A}^{-1}B + D \label{eq:gate46-1}\\
\myvec{sI-A} &= \myvec {s & -1 \\  1  & s+2} \label{eq:gate46-2}\\
\myvec{sI - A}^{-1} &= \frac{1}{s(s+2)+1} \myvec {s+2 & 1 \\ -1 & s}  \label{eq:gate46-3}
\end{align}

Referencing from equation \eqref{eq:gate46-3}, equation \eqref{eq:gate46-1} becomes 
\begin{align}
T.F &= \myvec{ \frac{3}{s^2+2s+1} & \frac{-2}{s^2+2s+1} } \myvec {s+2 & 1 \\ -1 & s} \myvec{0 \\ 1} + 1 \\
&= \myvec{ \frac{3}{s^2+2s+1} & \frac{-2}{s^2+2s+1} }  \myvec{1 \\ s} + 1 \\
&= \frac{s^2 + 4}{s^2 + 2s + 1} \\
H(s) &= T.F \\
H(s) &= \frac{s^2 + 4}{s^2+2s+1} \label{eq:gate46-8}
\end{align}

Substituting $s=j\omega$ in equation \eqref{eq:gate46-8},
\begin{align}
H(j\omega) &= \frac{4-(\omega)^2}{1+2j\omega-(\omega)^2}
\end{align}

Steady state output of system is zero:
\begin{align}
4-(\omega)^2 &= 0 \\
\omega &= 2 \text{ rad/sec}
\end{align}

\end{document}
