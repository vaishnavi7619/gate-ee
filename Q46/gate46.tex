\let\negmedspace\undefined
\let\negthickspace\undefined
\documentclass[a4,12pt,onecolumn]{IEEEtran}
\usepackage{amsmath,amssymb,amsfonts,amsthm}
\usepackage{algorithmic}
\usepackage{graphicx}
\usepackage{textcomp}
\usepackage{xcolor}
\usepackage{txfonts}
\usepackage{listings}
\usepackage{enumitem}
\usepackage{mathtools}
\usepackage{gensymb}
\usepackage[breaklinks=true]{hyperref}
\usepackage{tkz-euclide}
\usepackage{listings}
\usepackage{circuitikz}
\usepackage{gvv}
\begin{document}
\title{ signals and systems
\vspace{1cm}

Gate2023-ee-Q46}
\author{EE23BTECH11014- Devarakonda Guna vaishnavi}
\maketitle
\textbf{Question:}

Consider the state-space description of an LTI system with matrices
\begin{align}

A = \begin{pmatrix} \myvec{0} & \myvec{1} \\ \myvec{-1} & \myvec{-2}\end{pmatrix}, \quad 
B = \begin{pmatrix} \myvec{0} \\ \myvec{1} \end{pmatrix}, \quad 
C = \begin{pmatrix} \myvec{3} & \myvec{-2} \end{pmatrix}, \quad 
D = \begin{pmatrix} \myvec{1} \end{pmatrix}
\end{align}

For the input, $\sin(\omega t)$, $\omega > 0$, the value of $\omega$ for which the steady-state output of the system will be zero, is \underline{\hspace{2cm}} (Round off to the nearest integer).

solution:
\vspace{1cm}

\begin{table}[htbp]
    \centering
    \caption{Input Parameters}
    \label{tab:parameters}
    \begin{tabular}{|c|c|}
    \hline
    \textbf{Parameter} & \textbf{Value} \\
    \hline
    System Matrix, \(A\) & \(
    \begin{pmatrix}
    0 & 1 \\
    -1 & -2 \\
    \end{pmatrix}
    \) \\
    Input Matrix, \(B\) & \(
    \begin{pmatrix}
    0 \\
    1 \\
    \end{pmatrix}
    \) \\
    Output Matrix, \(C\) & \(
    \begin{pmatrix}
    3 & -2 \\
    \end{pmatrix}
    \) \\
    Feedthrough Matrix, \(D\) & \(1\) \\
    Input Signal, \(u(t)\) & \( \sin(\omega t) \), \( \omega > 0 \) \\
    \hline
    \end{tabular}
\end{table}


\vspace{2cm}

A = \begin{pmatrix} 
\myvec{0}& \myvec{1} \\ \myvec{-1} & \myvec{-2} 
\end{pmatrix}
B = \begin{pmatrix}
\myvec{0} \\ \myvec{1} 
\end{pmatrix}
C= \begin{pmatrix} \myvec{3} & \myvec{-2}\\ \end{pmatrix}
D= \begin{pmatrix} \myvec{1} \\  \end{pmatrix}
\vspace{1cm}

transfer function given by
\begin{align}
T.F = C(sI - A)^{-1}B + D
\end{align}
\vspace{1cm}




\begin{align}
\begin{pmatrix} \myvec{sI-A}  \\    \end{pmatrix}= \begin{pmatrix}\myvec{s} & \myvec{-1} \\ \myvec{1}  & \myvec{s+2}\end{pmatrix}
\begin{pmatrix} \myvec{sI - A }\\   \end{pmatrix}^{-1}=\begin{pmatrix}s & \myvec{-1}\\ \myvec{1} & \myvec{s+2} \end{pmatrix}^{-1}
\end{align}

\vspace{1cm}



\begin{align}
\begin{pmatrix} \myvec{sI - A} \\   \end{pmatrix}^{-1}=\frac{1}{s(s+2)+1}\begin{pmatrix} \myvec{s+2} & \myvec{1} \\ \myvec{-1} & \myvec{s}\end{pmatrix}\begin{pmatrix} \myvec{0} \\ \myvec{1} \end{pmatrix}
\end{align}
\vspace{1cm}

 from equation 3

\begin{align}
T.F=\begin{pmatrix} \myvec{3/s^2+2s+1} & \myvec{-2/s^2+2s+1} \\  \end{pmatrix}  
\begin{pmatrix} \myvec{s+2} & \myvec{1} \\ \myvec{-1} & \myvec{s}\end{pmatrix}\begin{pmatrix} \myvec{0}\\ \myvec{1} \end{pmatrix} +1
\end{align}  

solving equation 4 results

\vspace{1cm}
\begin{align}
T.F=\begin{pmatrix} \myvec(3/s^2+2s+1) & \myvec(-2/s^2+2s+1)  \\ \end{pmatrix}\begin{pmatrix}
\myvec{0}\\ \myvec{1} \end{pmatrix}+1
\end{align}
\vspace{1cm}

\begin{align}
T.F={s^2 + 4}*( 1/{s^2 + 2s + 1})\
\end{align}

\begin{align}
H(S)=T.F={s^2 + 4}*(\frac{1}{s^2+2s+1})
\end{align}
\vspace{3cm}

s=j\omega
\vspace{1cm}

substitute s=j\omega
  

\vspace{1cm}
\begin{align}
H(j\omega)={4-(\omega)^2}*(\frac{1}{1+2j\omega-(\omega)^2})
\end{align}
\vspace{1cm}


Steady state output of system is zero

\begin{align}
4-(\omega)^2=0
\end{align}
from equation 9


\omega=2rad/sec
\end{document}






